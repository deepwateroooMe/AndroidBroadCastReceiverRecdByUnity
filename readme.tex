% Created 2022-12-31 Sat 21:19
\documentclass[9pt, b5paper]{article}
\usepackage{xeCJK}
\usepackage{minted}
\usepackage[T1]{fontenc}
\usepackage[scaled]{beraserif}
\usepackage[scaled]{berasans}
\usepackage[scaled]{beramono}
\usepackage{graphicx}
\usepackage{xcolor}
\usepackage{multirow}
\usepackage{multicol}
\usepackage{float}
\usepackage{textcomp}
\usepackage{algorithm}
\usepackage{algorithmic}
\usepackage{latexsym}
\usepackage{natbib}
\usepackage{geometry}
\geometry{left=1.2cm,right=1.2cm,top=1.5cm,bottom=1.2cm}
\newminted{common-lisp}{fontsize=\footnotesize} 
\usepackage[xetex,colorlinks=true,CJKbookmarks=true,linkcolor=blue,urlcolor=blue,menucolor=blue]{hyperref}
\author{deepwaterooo}
\date{\today}
\title{游戏通用功能底层逻辑Android SDK 封装}
\hypersetup{
  pdfkeywords={},
  pdfsubject={},
  pdfcreator={Emacs 27.2 (Org mode 8.2.7c)}}
\begin{document}

\maketitle
\tableofcontents


\section{Unity BroadcastReceiver to receive android Broadcast}
\label{sec-1}
\begin{itemize}
\item 有几种不同的方法来实现:
\begin{itemize}
\item 已经实现了的接口的方式, 如:
\begin{itemize}
\item \url{https://github.com/deepwateroooMe/deepwateroooSDK/blob/master/dwsdk/src/main/java/com/deepwaterooo/sdk/utils/VoiceVolumnUtil.java}
\item \url{https://github.com/deepwateroooMe/deepwateroooSDK/blob/master/dwsdk/src/main/java/com/deepwaterooo/sdk/utils/VoiceVolumnChangedIntereface.java}
\item \url{https://github.com/deepwateroooMe/Tetris_Unity3D/blob/hotfixTrial/trunk/Assets/Scripts/deepwaterooo/tetris3d/SettingsCallback.cs}
\item 这种需要借助接口来实现安卓SDK端Broadcast接收后的回调到游戏端
\end{itemize}
\end{itemize}
\end{itemize}
   - 缺点是:代码偶合太严重,不方便系统性的源码解偶复用与移植
\begin{itemize}
\item 本质上同样也是通过公用(安卓SDk与游戏端)接口的形式,但是可以解偶,借助游戏端AnroidJavaProxy帮助
\begin{itemize}
\item 目前只实现了安卓SDK静态广播的接收
\item Todo: 游戏端是可以实现动态注册安卓广播,并可以发送广播的.这两块可以再测试加深一下
\item 主要参考: \url{https://blog.csdn.net/yhx956058885/article/details/110949067} 原贴的作者华人应该是想要迷惑众生(或者是想要我们去思考),所以它的贴子里埋了无数的bug. 以后也该明白当搜索不到中文来解决问题,就直接搜英文.因为中文能够搜索到的,是想要把你推向的方向;中文搜索不到的,才是自己真正需要钻研该走的路,爱表哥,爱生活!!!
\end{itemize}
\item 其它搜索到的思路与印记还包括:
\begin{itemize}
\item \url{https://www.cnblogs.com/alps/p/11206465.html} 也是接口的方式,没再细看
\item \url{http://jeanmeyblum.weebly.com/scripts--tutorials/communication-between-an-android-app-and-unity} 拐了拐
\item \url{https://zditect.com/code/intercall-between-unity-and-android.html}
\end{itemize}

\item 这个项目里大概有三种不同实现设计的源码,但目前只测试运行通了一种(源项目接口方式的一种, + 这里静态广播的第二种),自己项目中需要的关于音量变化的静态广播
\begin{itemize}
\item 没有删除其余待测试的,供自己改天再回来进一步地理解消化
\end{itemize}
\end{itemize}
% Emacs 27.2 (Org mode 8.2.7c)
\end{document}